\documentclass[a4paper, 10pt]{article}

\usepackage{url}
\usepackage{color}
\usepackage{listings}

\title{눈송이 서버}
\author{Alan Kim}

\begin{document}
\maketitle
\begin{abstract}
운영하면서 만들어지는 눈송이 서버들
\end{abstract}

% @suppress Contraction WeakExpression
\section{Summary}

서버 운영을 오래 해 본 사람이라도,처음 들어가는 서버에서는 마음 먹은 대로 문제를 해결하기가 어렵습니다. 이는 서버를 다루는 기술과는 별개로, 각 서버마다 운영 기록이 다르기 때문입니다. 똑같은 일을 하는 두 서버가 있다 해도, A 서버는 한 달 전에 구성했고 B 서버는 이제 막 구성했다면, 운영체제부터 컴파일러, 설치된 패키지까지 완벽하게 같기는 쉽지 않습니다. 그리고 이러한 차이점들이 장애를 일으키고 말죠. 'A 서버는 잘 되는데 B 서버는 왜 죽었지?"와 같은 일 (혹은 그 반대)이 벌어지는 겁니다.
이렇게 서로 모양이 다른 서버들이 존재하는 상황을 눈송이 서버(Snowflakes Server)라고도 합니다. 모든 눈송이의 모양이 다르듯, 서버들도 서로 다른 모습이라는 말이죠.

%% \bibliographystyle{plain}
%% \bibliography{reference}

\end{document}
